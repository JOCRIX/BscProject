%page setup

\documentclass[12pt,twoside,a4paper,openright]{report}
%asd
%unicode encoding


\usepackage[utf8]{inputenc}

%For thick vector bar over a character
%\usepackage{accents}

% Make latex understand and use the typographic
% rules of the language used in the document.
\usepackage[danish,english]{babel}
% Context Sensitive Quotation
\usepackage{csquotes}
% Use the palatino font
\usepackage[sc]{mathpazo}
\linespread{1.05}         % Palatino needs more leading (space between lines)

% Choose the font encoding
\usepackage[T1]{fontenc}
\usepackage{setspace} %Gives the possiblity to edit line spacing.
\usepackage[misc,clock,geometry]{ifsym}

%Laplace 
\usepackage[scr]{rsfso}

% load a colour package
\usepackage{xcolor}
\definecolor{aaublue}{RGB}{33,26,82}% dark blue
% The standard graphics inclusion package
\usepackage{graphicx}
\usepackage{subfig}

% Set up how figure and table captions are displayed
\usepackage{float}

%%%%%%%%%%%%%%%%%%%%%%%%%%%%%%%%%%%%%%%%%%%%%%%%
% Mathematics
% http://en.wikibooks.org/wiki/LaTeX/Mathematics
%%%%%%%%%%%%%%%%%%%%%%%%%%%%%%%%%%%%%%%%%%%%%%%%
% Defines new environments such as equation,
% align and split 
\usepackage{amsmath}
\usepackage[makeroom]{cancel}
% Adds new math symbols
\usepackage{amssymb}
% Use theorems in your document
% The ntheorem package is also used for the example environment
% When using thmmarks, amsmath must be an option as well. Otherwise \eqref doesn't work anymore.
\usepackage[framed,amsmath,thmmarks]{ntheorem}

\usepackage{changepage}
% Change margins, papersize, etc of the document
\usepackage[
  showframe=false, %Shows margins on pages if necessary
  inner=25mm,% left margin on an odd page
  outer=25mm,% right margin on an odd page
  top=20mm,% Top margin on an odd page
  ]{geometry}
\usepackage{caption}
\captionsetup{%
  font=footnotesize,% set font size to footnotesize
  labelfont=bf % bold label (e.g., Figure 3.2) font
}

% Make the standard latex tables look so much better
\usepackage{array,booktabs}

% Enable the use of frames around, e.g., theorems
% The framed package is used in the example environment
\usepackage{framed}
\usepackage{wrapfig}
%\usepackage{background}
\usepackage{amsmath, amsfonts, amssymb}


%long tables enable page wide tables
\usepackage{longtable}
%array for aligning text in table
\usepackage{array}
%svg figures
\usepackage{svg}
\usepackage{tikz}
%\usepackage{ctable}



%%%%%%%%%%%%%%%%%%%%%%%%%%%%%%%%%%%%%%%%%%%%%%%%
% Mathematics
% http://en.wikibooks.org/wiki/LaTeX/Mathematics
%%%%%%%%%%%%%%%%%%%%%%%%%%%%%%%%%%%%%%%%%%%%%%%%
% Defines new environments such as equation,
% align and split 
\usepackage{amsmath}
\usepackage[makeroom]{cancel}
% Adds new math symbols
\usepackage{amssymb}
\usepackage{textcomp, gensymb}
% Use theorems in your document
% The ntheorem package is also used for the example environment
% When using thmmarks, amsmath must be an option as well. Otherwise \eqref doesn't work anymore.
\usepackage[framed,amsmath,thmmarks]{ntheorem}

\usepackage{changepage}
% Change margins, papersize, etc of the document
\usepackage[
  showframe=false, %Shows margins on pages if necessary
  inner=25mm,% left margin on an odd page
  outer=25mm,% right margin on an odd page
  top=20mm,% Top margin on an odd page
  ]{geometry}

  \usepackage{titlesec}
\titleformat{\chapter}[hang]{\normalfont\Huge\bfseries}{\thechapter: }{0pt}{\Huge}
\titleformat*{\section}{\normalfont\Large\bfseries}
\titleformat*{\subsection}{\normalfont\large\bfseries}
\titleformat*{\subsubsection}{\normalfont\normalsize\bfseries}
\titlespacing*{\chapter}{0pt}{-30pt}{40pt}
%\titleformat*{\paragraph}{\normalfont\normalsize\bfseries}
%\titleformat*{\subparagraph}{\normalfont\normalsize\bfseries}
\setlength{\parindent}{0pt}
\setlength{\headheight}{14.49998pt}
%afstand mellem kapitel og toppen af siden
\titlespacing*{\chapter}{0pt}{-40pt}{40pt}

% Clear empty pages between chapters
\let\origdoublepage\cleardoublepage
\newcommand{\clearemptydoublepage}{%
  \clearpage
  {\pagestyle{empty}\origdoublepage}%
}
\let\cleardoublepage\clearpage

% Change the headers and footers
\usepackage{fancyhdr}
\pagestyle{fancy}
\fancyhf{} %delete everything
\renewcommand{\headrulewidth}{0pt} %remove the horizontal line in the header
\fancyhead[RE]{\small\nouppercase\leftmark} %even page - chapter title
\fancyhead[LO]{\small\nouppercase\rightmark} %uneven page - section title
\fancyhead[LE,RO]{\thepage} %page number on all pages
% Do not stretch the content of a page. Instead,
% insert white space at the bottom of the page
\raggedbottom
% Enable arithmetics with length. Useful when
% typesetting the layout.
\usepackage{calc}
%setup of code snippets
\usepackage{listings}

\definecolor{gainsboro}{rgb}{0.86, 0.86, 0.86}
\definecolor{jokkesgraa}{rgb}{0.9490196078431372,0.9490196078431372,0.9215686274509803}
\lstdefinestyle{MatLab}{ %Formatting for code in appendix
    language=Matlab,
    numbers=left,
    frame=L,
    stepnumber=1,
    showstringspaces=false,
    tabsize=1,
    breaklines=true,
    breakatwhitespace=false,
    commentstyle=\itshape\color{green!40!black},
    keywordstyle=\bfseries\color{blue},
    backgroundcolor = \color{jokkesgraa},
}
\lstdefinestyle{C}{ %Formatting for code in appendix
    language=C, 
    numbers=left,
    frame=L,
    stepnumber=1,
    showstringspaces=false,
    tabsize=1,
    breaklines=true,
    breakatwhitespace=false,
    commentstyle=\itshape\color{green!40!black},
    keywordstyle=\color{blue},
    stringstyle=\color{orange!70!black},
    backgroundcolor = \color{jokkesgraa},
}
% Er lige ved at se om vi skal have en ekstra package for at definere farver
%prøv den her color code #f2f2eb alright alright
%%%%%%%%%%%%%%%%%%%%%%%%%%%%%%%%%%%%%%%%%%%%%%%%
% Bibliography
% http://en.wikibooks.org/wiki/LaTeX/Bibliography_Management
%%%%%%%%%%%%%%%%%%%%%%%%%%%%%%%%%%%%%%%%%%%%%%%%
\usepackage[nottoc]{tocbibind}
\usepackage[urldate=long,backend=biber,style=numeric,dateabbrev=false,sortcites,natbib=true,sorting=none]{biblatex}
\renewbibmacro*{urldate}{
(visited on \printfield{urlday}/\printfield{urlmonth}/\printfield{urlyear})
}
\addbibresource{bib/mybib.bib}%add bibliography as resource

%%%%%%%%%%%%%%%%%%%%%%%%%%%%%%%%%%%%%%%%%%%%%%%%
% Misc
%%%%%%%%%%%%%%%%%%%%%%%%%%%%%%%%%%%%%%%%%%%%%%%%
% Add bibliography and index to the table of
% contents
% Add the command \pageref{LastPage} which refers to the
% page number of the last page
\usepackage{lastpage}

% Add todo notes in the margin of the document
\usepackage[
  disable, %turn off todonotes
  colorinlistoftodos, %enable a coloured square in the list of todos
  textwidth=0.7\marginparwidth, %set the width of the todonotes
  textsize=scriptsize, %size of the text in the todonotes
  ]{todonotes}
\usepackage{parskip}
\setlength{\parskip}{1em}

\usepackage{hyperref}
\hypersetup{%
	plainpages=false,%
	pdfauthor={Gustav R T, Jakob T, Joachim A, Kasper Rødtnes, Thorvald Einarson},%
	pdftitle={P7, characterizing Passiv Components},%
	pdfsubject={Semester project},%
	bookmarksnumbered=true,%
	colorlinks=false,%
	citecolor=grey,%
	filecolor=black,%
	linkcolor=black,% you should probably change this to black before printing
	urlcolor=blue,%
	pdfstartview=FitH%
}
\usepackage{blindtext}
\usepackage{booktabs}
\usepackage{subcaption}
\usepackage{pdfpages}
\usepackage{colortbl}
\usepackage{booktabs,caption}
\usepackage[flushleft]{threeparttable}


\usepackage{titlesec}
\setcounter{secnumdepth}{4}

\titleformat{\paragraph}
{\normalfont\normalsize\bfseries}{\theparagraph}{1em}{}
\titlespacing*{\paragraph}
{0pt}{3.25ex plus 1ex minus .2ex}{1.5ex plus .2ex}

%Beautiful conditions
\usepackage{tabularx}
\newlength{\conditionwd}
\newenvironment{conditions}[1][where:]
  {%
   #1\tabularx{\textwidth-\widthof{#1}}[t]{
     >{$}l<{$} @{${}={}$} X@{}
   }%
  }
  {\endtabularx\\[\belowdisplayskip]}

%Beautiful conditions continiued
\newlength{\belongswd}
\newenvironment{belongs}[1][where:]
  {%
   #1\tabularx{\textwidth-\widthof{#1}}[t]{
     >{$}l<{$} @{${}\in {}$} X@{}
   }%
  }
  {\endtabularx\\[\belowdisplayskip]}

%\renewcommand*\contentsname{Summary} %Changes the name of the contents list
\usepackage[export]{adjustbox}
\usepackage{siunitx} % units
%\sisetup{}
%\sisetup{exponent-mode = engineering}
\sisetup{per-mode = symbol}
\usepackage{makecell}

\usepackage{subcaption}

\usepackage[super]{nth}



%  A simple AAU report template.
%  2015-05-08 v. 1.2.0
%  Copyright 2010-2015 by Jesper Kjær Nielsen <jkn@es.aau.dk>
%
%  This is free software: you can redistribute it and/or modify
%  it under the terms of the GNU General Public License as published by
%  the Free Software Foundation, either version 3 of the License, or
%  (at your option) any later version.
%
%  This is distributed in the hope that it will be useful,
%  but WITHOUT ANY WARRANTY; without even the implied warranty of
%  MERCHANTABILITY or FITNESS FOR A PARTICULAR PURPOSE.  See the
%  GNU General Public License for more details.
%
%  You can find the GNU General Public License at <http://www.gnu.org/licenses/>.
%
%
%
% see, e.g., http://en.wikibooks.org/wiki/LaTeX/Customizing_LaTeX#New_commands
% for more information on how to create macros

\newcommand{\e}{\mathrm{e}}
\newcommand{\mconst}[1]{\mathrm{#1}}

%%%%%%%%%%%%%%%%%%%%%%%%%%%%%%%%%%%%%%%%%%%%%%%%
% Macros for the titlepage
%%%%%%%%%%%%%%%%%%%%%%%%%%%%%%%%%%%%%%%%%%%%%%%%
%Creates the aau titlepage
\newcommand{\aautitlepage}[3]{%
  {
    %set up various length
    \ifx\titlepageleftcolumnwidth\undefined
      \newlength{\titlepageleftcolumnwidth}
      \newlength{\titlepagerightcolumnwidth}
    \fi
    \setlength{\titlepageleftcolumnwidth}{0.5\textwidth-\tabcolsep}
    \setlength{\titlepagerightcolumnwidth}{\textwidth-2\tabcolsep-\titlepageleftcolumnwidth}
    %create title page
    \thispagestyle{empty}
    \noindent%
    \begin{tabular}{@{}ll@{}}
      \parbox{\titlepageleftcolumnwidth}{
        \iflanguage{danish}{%
          \includegraphics[width=\titlepageleftcolumnwidth]{aaugraphics/aau_logo_da.pdf}
        }{%
          \includegraphics[width=\titlepageleftcolumnwidth]{aaugraphics/aau_logo_en.pdf}
        }
      } &
      \parbox{\titlepagerightcolumnwidth}{\raggedleft\sf\small
        #2
      }\bigskip\\
       #1 &
      \parbox[t]{\titlepagerightcolumnwidth}{%
      \textbf{Abstract:}\bigskip\par
        \fbox{\parbox{\titlepagerightcolumnwidth-2\fboxsep-2\fboxrule}{%
          #3
        }}
      }\\
    \end{tabular}
    \vfill
    \clearpage
  }
}

%Create english project info
\newcommand{\englishprojectinfo}[9]{%
  \parbox[t]{\titlepageleftcolumnwidth}{
    \textbf{Title:}\\ #1\bigskip\par
    \textbf{Theme:}\\ #2\bigskip\par
    \textbf{Project Period:}\\ #3\bigskip\par
    \textbf{Project Group:}\\ #4\bigskip\par
    \textbf{Participants:}\\ #5\bigskip\par
    \textbf{Supervisors:}\\ #6\bigskip\par
    \textbf{Copies:} #7\bigskip\par
    \textbf{Page Numbers:} #8 \\\bigskip\par
    \textbf{Date of Completion:}\\ #9
  }
}

%Create danish project info
\newcommand{\danishprojectinfo}[8]{%
  \parbox[t]{\titlepageleftcolumnwidth}{
    \textbf{Titel:}\\ #1\bigskip\par
    \textbf{Tema:}\\ #2\bigskip\par
    \textbf{Projektperiode:}\\ #3\bigskip\par
    \textbf{Projektgruppe:}\\ #4\bigskip\par
    \textbf{Deltager(e):}\\ #5\bigskip\par
    \textbf{Vejleder(e):}\\ #6\bigskip\par
    \textbf{Oplagstal:} #7\bigskip\par
    \textbf{Sidetal:} \pageref{LastPage}\bigskip\par
    \textbf{Afleveringsdato:}\\ #8
  }
}








%%%%%%%%%%%%%%%%%%%%%%%%%%%%%%%%%%%%%%%%%%%%%%%%
% An example environment
%%%%%%%%%%%%%%%%%%%%%%%%%%%%%%%%%%%%%%%%%%%%%%%%
\theoremheaderfont{\normalfont\bfseries}
\theorembodyfont{\normalfont}
\theoremstyle{break}
\def\theoremframecommand{{\color{gray!50}\vrule width 5pt \hspace{5pt}}}
\newshadedtheorem{exa}{Example}[chapter]
\newenvironment{example}[1]{%
		\begin{exa}[#1]
}{%
		\end{exa}
}
%%%%%%%%%%%%%%%%%%%%%%%%%%%%%%%%%%%%%%%%%%%%%%%%%%%%%%%%%%
%Macros
%%%%%%%%%%%%%%%%%%%%%%%%%%%%%%%%%%%%%%%%%%%%%%%%%%%%%%%%%%

\newcommand{\es}{\\[20pt]}

\newcommand{\subsubsubsection}[1]{
\paragraph{#1}
}

%Test command til at lave kommandoer med optional arguments.
%nr. 4 viser hvordan man kan lave en default hvis den er efterladt tom.
\newcommand{\test}[4][]{
\ifthenelse{\equal{#1}{}}{}{#1 \\[20pt]}
\ifthenelse{\equal{#2}{}}{}{#2 \\[20pt]}
\ifthenelse{\equal{#3}{}}{}{#3 \\[20pt]}
\ifthenelse{\equal{#4}{}}{default value}{#4}
}

\newcommand{\AAU}[1][]{AAU$^3$ }

\DeclareSIUnit{\kmph}{\kilo \meter\per\hour}

%\test[XXX]{AAA}{BBB}{CCC}
%Til PDF i appendix
\newcommand{\pdfappendix}[3][]{
\includepdf[pages=1,scale=1,pagecommand={\subsection{#1}\label{#2}},linktodoc=true]{#3}
\includepdf[pages=2-,scale=1,linktodoc=true]{#3}
}
%Til Chapter
\newcommand{\Chap}[2]{\chapter{#1}\label{ch:#2}}

%Dette gøres ved at skrive \Chap{}{}, hvor den første er Titlen og den anden er label.
%Til Section
\newcommand{\Sec}[2]{\section{#1}\label{sec:#2}}

%Til SubSection
\newcommand{\Sub}[2]{\subsection{#1}\label{subsec:#2}}

%Til SubsubSection
\newcommand{\Subsub}[2]{\subsubsection{#1}\label{subsubsec:#2}}
%%%%%%%%%%%%%%%%%%%%%%%%%%%%%%%%%%%%%%%%%%%%%%%%












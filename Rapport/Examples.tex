


\section{References, labels and abbreviations}
% Please add the following required packages to your document preamble:
% \usepackage[table,xcdraw]{xcolor}
% If you use beamer only pass "xcolor=table" option, i.e. \documentclass[xcolor=table]{beamer}
\begin{table}[H]
\begin{tabular}{|l|l|}
\hline

Section type  & Reference/Label name \\ \hline
Chapter       & ch:name                  \\ \hline
Section       & sec:name                  \\ \hline
Subsection    & subsec:name               \\ \hline
Subsubsection & subsubsec:name           \\ \hline
Figure        & fig:name                  \\ \hline
Table         & tab:name                  \\ \hline
Equation      & eq:name                   \\ \hline
code snippets & snip:name                 \\ \hline
Appendix      & app:name                  \\ \hline
Bibliography  & bib:name\_addition\_another\_addition                \\ \hline
\end{tabular}
\end{table}
\section{Figures and tables}
\subsection{2 by x table}
\begin{table}[H]
\begin{tabular}{|l|l|}
\hline

        &       \\ \hline
        &       \\ \hline
        &       \\ \hline
\end{tabular}
\end{table}

\subsection{3 by x table}
\begin{table}[H]
\begin{tabular}{|l|l|l|}
\hline

        &       &       \\ \hline
        &       &       \\ \hline
        &       &       \\ \hline
\end{tabular}
\end{table}
\subsection{colored 3 by x table}
\begin{table}[H]
\begin{tabular}{|l|l|l|}
\hline
 
&  &  \\ \hline
 &  &  \\ \hline
 &  &  \\ \hline
\end{tabular}
\end{table}

\subsection{Figure figure}
Inserting figures of type: PDF, JPEG, JPG, PNG
\begin{figure}[H]
    \centering
    \includegraphics[width=0.5\textwidth]{aaugraphics/aau_logo_circle_en.pdf}
    \caption{Caption}
    \label{fig:enter_label2}
\end{figure}

\subsection{SVG figure}
\begin{figure}[H]
    \centering
    \includesvg[width=0.3\textwidth]{Sections/1_Introduction/figures/Smiley.svg}
    \caption{ SVG smiley. He is happy be cause the file size is small but the resolution is high}
    \label{fig:enter_label1}
\end{figure}

\section{Unit and equations}
Examples of how to use units and different equations
\subsection{SI units}

\subsection{Equations}
\paragraph*{numerated equation:}
In equation \ref{ex:eq1} how to set up a enumerated equation can be seen.
\begin{equation}\label{ex:eq1}
    K_T = \frac{\tau}{I_a}
\end{equation}
\begin{conditions}
    K_T     & motor constant\\
    \tau    & the torque the motor produces\\
    I_a     & the current the motor draws\\
\end{conditions}


\paragraph*{unnumerated equation:}
In equation \ref{ex:eq2} how to set up an unenumerated equation can be seen. The reference however is changed to the nearest reference-able title object. Always try to add a reference to the equation, it might come in handy.
\begin{equation*}\label{ex:eq2}
    K_T = \frac{\tau}{I_a}
\end{equation*}
\begin{conditions}
    K_T     & motor constant\\
    \tau    & the torque the motor produces\\
    I_a     & the current the motor draws\\
\end{conditions}

\paragraph*{How to do math in latex}
If the table below is not fulfilling:
\url{https://www.cmor-faculty.rice.edu/~heinken/latex/symbols.pdf}

\begin{table}[H]
\centering
\renewcommand{\arraystretch}{1.5}
\begin{tabular}{|l |c|l|} \hline   
addition & + & + \\ \hline 
subtraction & - & - \\ \hline 
multiplication & $\cdot$ & \textbackslash{cdot} \\ \hline 
division & $\frac{num}{denom}$ & \textbackslash{frac\{num\}\{denom\}} \\ \hline 
powers & $a^{b+c}$ & a\^ \space\{b+c\}\\ \hline 
square root & $\sqrt{a+b} $ & \textbackslash{sqrt(a+b)} \\ \hline 
summation & $\sum $ & \textbackslash{sum} \\ \hline 
integration & $\int$ & \textbackslash{}int \\ \hline 
2x int & $\iint$ & \textbackslash{}iint \\ \hline 
3x int & $\iiint$ & \textbackslash{}iiint \\ \hline 
differentiation & $\dot{a}$ & \textbackslash{Ddot\{a\{}\\ \hline 
2x diff & $\Ddot{a}$ & \textbackslash{Ddot\{a\{}\\ \hline 
vector notation & $\overline{\rm AB}$  & \textbackslash{}overline\{\rm AB\}  \\ \hline

\end{tabular}

\end{table}


\chapter{DAC Passive Reconstruction Filter} \label{App:DAC_PASSIVE_FILTER}
This appendix will show the method used to estimate the passive DAC filter. 

The pass-band $f_p$ is desired to be at \SIQ{1.5}{\mega\hertz}, this results in a frequency scaling factor $k_f$ of $k_f = 2\pi\cdot1.5e6 = 9.4248e6$. The filter is designed such that the load impedance of the filter is the \SIQ{50}{\ohm} resistor used to convert the DAC current to a voltage. This indicates that the impedance scalign factor is $k_z = 50$.

Using nothing more than these two values, a filter can be designed by a standard LC table for a normalized low-pass Butterworth filter such as shown in figure \ref{fig:app_DAC_LC_TABLE}. A 4th order filter sets the normalized values to $L_1 = 1.5307$, $C_2 = 1.5772$, $L_3 = 1.0824$, $C_4 = 0.3827$.

\begin{figure}[H]
    \centering
    \includegraphics[clip, trim=0 0 0 0, width=1\textwidth]{Appendix/Figures/Butterworth_LC.pdf}
    \caption{Standard normalized low-pass Butterworth LC table.}
    \label{fig:app_DAC_LC_TABLE}
\end{figure}

Using the frequency and impedance scaling factor, component values can then be found. This can be seen in equation \ref{eq:app_DAC_1}.
\begin{equation}
\label{eq:app_DAC_1}
\begin{split}
    L_1 &= \frac{k_z}{k_f}1.5307 \Rightarrow L_1 = \frac{50}{9.4248e6}1.5307 = \SIQ{8.12}{\micro\henry}\\
    C_2 &= \frac{1}{k_z k_f}1.5772 \Rightarrow C_2 = \frac{1}{50\cdot9.4248e6}1.5772 = \SIQ{3347}{\pico\farad} \\
    L_3 &= \frac{k_z}{k_f}1.0824 \rightarrow L_3 = \frac{50}{9.4248e6}1.0824 = \SIQ{5.74}{\micro\henry}\\
    C_4 &= \frac{1}{k_z k_f}0.3827 \Rightarrow C_2 = \frac{1}{50\cdot9.4248e6}0.3827 = \SIQ{812}{\pico\farad} \\
\end{split}
\end{equation}

These exact values where not available, the used values are as in equation \ref{eq:app_DAC_2}.
\begin{equation}
    \label{eq:app_DAC_2}
    \begin{split}
        L_1 &= \SIQ{8.2}{\micro\henry}\\
        C_2 &= \SIQ{3300}{\pico\farad} \\
        L_3 &= \SIQ{5.6}{\micro\henry}\\
        C_4 &= \SIQ{820}{\pico\farad} \\
    \end{split}
    \end{equation}

To analyze the behaviour of the Butterworth filter, the poles of the filter have also been found. The poles of a Butterworth filter can be found as in equation \ref{eq:app_DAC_3}.

\begin{equation}
\label{eq:app_DAC_3}
\begin{split}
    p_k &= \e^{j\left(\frac{2k\pi-\pi}{2n}+\frac{\pi}{2}\right)} \\
    p_1 &= -0.3827+j0.9239 \\
    p_2 &= -0.9239+j0.3827 \\
    p_3 &= -0.9239-j0.3827 \\
    p_4 &= -0.3827-j0.9239 \\
    k &= 1, \:2,\:3\:4 \\
    n &= 4
\end{split}
\end{equation}

These poles can then be scaled by the frequency scalign factor and combined with their complex conjugate counter part. This can be seen in equation \ref{eq:app_DAC_4}. The transfer function of this Butterworth filter can also be seen.

\begin{equation}
\label{eq:app_DAC_4}
\begin{split}
    p_a &= (s-p_1 k_f)(s-p_4 k_f) = s^2 +7.2134e6 \:s+8.8826e13 \\
    p_b &= (s-p_2 k_f)(s-p_3 k_f) = s^2 +1.7415e7 \:s+8.8826e13 \\
    \\
    H(s) &= \frac{{k_f}^4}{p_a p_b} \Rightarrow \frac{7.8901e27}{(s^2 +7.2134e6 \:s+8.8826e13)(s^2 +1.7415e7 \:s+8.8826e13)}
\end{split}
\end{equation}


\subsection{Pulse width modulated pulse generator} \label{subsec:PWMGen} 

This pulse generator is a more advanced version of the generator in section \refq{subsec:Pulse_Generator}. The generator will produce \textit{N} amount of pulses and provides control over the widths of the high and low time of each pulse. The SPI CLK signal used to interface the FPGA to the LTC2311 ADC and described in section \refq{subsec:ADC_CONTROL} requires that the CLK signal is logic high whenever it is not active, so this has also been done with this generator.
The full code for this pulse generator can be seen in appendix \refq{App:PWMGenCode}, but the heart of the generator is two state machines that will be explained in this section.

The pulse generator will generate it's clock pulses based on the MASTER\_CLK input. The CLK could be anything, but it is assumed to be \SIQ{200}{\mega\hertz} in this project and any pulse width will thus be a multiple of the period of a \SIQ{200}{\mega\hertz} signal, which is \SIQ{5}{\nano\second}.

This generator is based on counters for producing accurate 'high' and 'low' periods. The code for generating a single pulse can be seen on listing \refq{lst:7_2_9_PWM_GEN_1_PULSE}.
\lstinputlisting[language=C ,style = c,firstnumber=1, linerange=128-164, caption={VHDL code for generating a single pulse.}, label={lst:7_2_9_PWM_GEN_1_PULSE}]{Sections/7_SystemDesign/Code/pulse_gen_width_modulator_invert.vhd}. 

The process on listing \refq{lst:7_2_9_PWM_GEN_1_PULSE} will generate a single pulse on the on the 'output\_set' signal. If the 'gen\_1\_pulse' signal is asserted and 'pulse\_complete' is cleared then it will proceed to generate a pulse. As can be seen in lines 7 - 27 the generator is a state machine that alternates between a 'HIGH' case and a 'LOW' case. The default case is 'HIGH'.

During the 'HIGH' case; the output is set to '1' and on every rising edge of MASTER\_CLK the counter inside the 'HIGH' case will increment 'clk\_count' in line 11 until it reaches the 'high\_clks' limit. 'high\_clks' is a parameter that can be set during the instantiation of the entity, see appendix \refq{App:PWMGenCode} for the full code. Once the counter has reached the 'high\_clks' value the state will be changed to 'LOW' in lines 12-15 and the process will start timing the 'LOW' period of the output signal.

The 'LOW' case functions exactly like the 'HIGH' case. The output is set to logical '0' and a counter will increment until it reaches the 'low\_clks' value. When it reaches the 'low\_clks' value the state will change back to 'HIGH' and a 'pulse\_complete' signal will be asserted which lets the process in listing \refq{lst:7_2_9_PWM_GEN_SEVERAL_PULSES} know that it it has finished a pulse and is ready for the next one.

\lstinputlisting[language=C ,style = c,firstnumber=1, linerange=91-117, caption={VHDL code for generating multiple pulses}, label={lst:7_2_9_PWM_GEN_SEVERAL_PULSES}]{Sections/7_SystemDesign/Code/pulse_gen_width_modulator_invert.vhd}. 

The VHDL code listing in \refq{lst:7_2_9_PWM_GEN_SEVERAL_PULSES} will, when triggered by an external source, start producing pulses until it reaches the 'NR\_OF\_CLKs' value. When triggered it will assert the 'gen\_1\_pulse' flag and this will start the generation of a single pulse in the way described in listing \refq{lst:7_2_9_PWM_GEN_1_PULSE}. When a single pulse is completed 'gen\_1\_pulse' is released and the process will re-assert 'gen\_1\_pulse' in order to generate another pulse. It continous in this way until the pulse generator has generated 'NR\_OF\_CLKs'.

The pulse generator also has an active/busy signal output that can be used to check if the generator is done pulsing, this can be seen in listing \refq{lst:7_2_9_PWM_GEN_ACTIVE}.

\lstinputlisting[language=C ,style = c,firstnumber=1, linerange=119-126, caption={VHDL code for controlling the active flag.}, label={lst:7_2_9_PWM_GEN_ACTIVE}]{Sections/7_SystemDesign/Code/pulse_gen_width_modulator_invert.vhd}. 

The active flag will set 'active = 1' when the pulse generator gets triggered på an external module. The active flag gets cleared when the pulse generator has reached 'NR\_OF\_CLKs' as seen in line 22 of listing \refq{lst:7_2_9_PWM_GEN_SEVERAL_PULSES}. This module has been tested and is used to generate the SPI CLK signal in the ADC control module and these measurements can be seen in appendix \refq{App:ADCControlMeasurement}.






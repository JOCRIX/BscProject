\section{MCU and Sample Control communication} \label{subsec:MCUSCCommunication} 
This section will focus on how the main processing unit implements and handles the communication protocol with the sample control module. 
The protocol designed for communication between the main processing unit and the Sample Control Unit has previously been described in section \refq{subsec:Communication}.

C is not an object oriented language, the principle of objects is however the focal point of the implementation of the communication code. This is implemented through the struct CommPort of the designated data-type CommunicationPort. This data-type holds 3 main data-types, \textbf{Control}, \textbf{Set} and \textbf{Get}, respectivly wtih the names \textbf{ctrl}, \textbf{set} and \textbf{get}. This can be seen in listing \refq{lst:7_3_1_StructCommPort}.

\lstinputlisting[language=c ,style = c,firstnumber=1, linerange=175-202, caption={The CommPort struct of the CommunicationPort data-type used to control the communication port with the Sample Control Module.}, label={lst:7_3_1_StructCommPort}]{Appendix/Code/main_2612.tex}

These 3 stucts are used to control the pins, ports etc. that the port is physically connected to. This can be used to set up the hardware. \textbf{Set} is used to set the port, such as setting an output value on the port, configuring it as inputs etc. \textbf{Get} is the general struct that will be used, as it utilizes call-back functions using the struct \textbf{Set} and \textbf{Control}. In this way, the hardware must be set up once, hereafter the port can be completly controlled by the \textbf{Get} struct call-back functions. 

Pointers and call-back functions have been extensively used in the CommPort struct. This has been a point of focus, as it makes it very intuitive and easy for the programmer to set up a register on the Sample Control Module. For example, if one wishes to set up the Sample Control Module for a total sample size of 10000, one would simply have to first set the IX-pin for internal, and then send the value 10000 to register 6. This can be done by using the CommPort struct as:

\begin{itemize}
    \item CommPort.set.SetIXMode(INTERNAL);
    \item CommPort.WriteData(10000, 6);
\end{itemize}

Note that INTERNAL here belongs to an enum with, INTERNAL = 0 and EXTERNAL = 1. In this way one does not have to keep track of the ones and zeros, but rather the functionality instead.

To fully describe the system, the handling of the protocol will be described in more detail specifically for the Control part, Set part and Get part. Not all functions will be described here, only those that are key for the opperation, as many are very similar.

\subsection*{Control data-type}
In order for the call-back functions to function, all the hardware must be specified to the CommPort struct. This is done through the \textbf{Control} data-type in listing \refq{lst:7_3_1_StructCommPort}. The specific pins can be found in appendix \refq{App:PinMap_MCU_FPGA}, the setup of these pins can be seen in listing \refq{lst:7_3_1_CommPort_PinSetup}, note that line 1 indicates that it points to its own address, allowing the struct to access data inside the struct itself.

\lstinputlisting[language=c ,style = c,firstnumber=1, linerange=1619-1643, caption={Pin setup for the CommPort struct.}, label={lst:7_3_1_CommPort_PinSetup}]{Appendix/Code/main_2612.tex}

This way of setting up the port makes it very easy to change pins or swap processor if needed. Here the used pins need only be reconfigured in the setup of the struct. This also comes in handy when opperating with an ARM processor such as the STM32F446RE. When writing to or reading from a pin, one must both specify the pin and the specific port of that pin. The task of keeping track of the port of each pin quickly becomes a dreary one, this is simply avoided by the given implementation.


\subsection*{Set data-type}
The \textbf{Set} struct of the CommunicationPort data-type is a collection of call-back functions specifically used for setting up the port. The different options within this struct are as in line 14 to 20 in listing \refq{lst:7_3_1_CommPort_PinSetup}. Here only \textbf{SetIOMode}, \textbf{SetIOValue} and \textbf{GetIOValue} will be fully described. The other functions are fairly simple.

Many of these call-back functions takes in enumerators as inputs, these simply act as a way for the human programmer to keep track of the ones and zeros. The \textbf{IXMode} enum can take either the value INTERNAL = 0 or EXTENRAL = 1. It is however much easier to keep track of the functionality than it is to keep track of a zero and one. Thus if it is desired to set the IX pin to internal, one would simply call the function as \textit{CommPort.Set.SetIOMode(INTERNAL)}, the compiler will then take care of inserting a 0 instead of the text INTERNAL. 

All these functions return an 8 bit signed integer, this is a status variable, if the function is executed as expected the return value will be a 1, if the hardware in the \textbf{ctrl} struct have not been set-up such that the CommPort.ctrl.selfAddr is a null pointer, all functions will return a -1. This is done, as operating on a null pointer will have the system crash, so the pointer is first sanity checked. If it is a null pointer, the function is not executed and an error flag is returned.

All functions in the \textbf{set} struct are call-back functions, the specifc functions that these point to can be seen in listing \refq{lst:7_3_1_CommPort_setPointers}.
\lstinputlisting[language=c ,style = c,firstnumber=1, linerange=1609-1615, caption={Directing the call-back functions to the specific funtion that they should carry out when called upon.}, label={lst:7_3_1_CommPort_setPointers}]{Appendix/Code/main_2612.tex}

\subsection*{SetGPIOMode}
The \textbf{SetGPIOMode} function can be seen in listing \refq{lst:7_3_1_SetIOMode}. This function configures the port for either an input or output. The port consists of 16 pins, pins 0-7 on port B and pins 0-7 on port C of the STM32F446RE. Port B pins 0-7 is bit 0-7 of the communication port and port, port C pins 0-7 is bit 8-15 of the communication port.
\lstinputlisting[language=c ,style = c,firstnumber=1, linerange=1234-1282, caption={Code for the SetGPIOMode function used to the 16 bit wide port to either and input or ourput.}, label={lst:7_3_1_SetIOMode}]{Appendix/Code/main_2612.tex}

The enumerator \textbf{enum IOMode} is used for input of the function can either take the value READ or WRITE. The \textbf{static GPIO\_InitTypeDef} in line 4 and 5 is an STM32 struct used to set up the hardware of a pin/port. It configures the port mode, speed and if pullup or pulldown resistors should be used.

The function essentially works by loading in the configured hardware from the \textbf{ctrl} struct, line 3, if the function is then called to configure the port for READ mode i.e. an input, it then loads in the specific hardware pins to the HighByte and LowByte struct by a loop, line 10 to 14. With all the pins loaded in to the STM32 struct, GPIO\_InitTypeDef, for the high and low byte, the port is simply configured for an input, line 18 and 19, for pull-up resistors, line 20 and 21, and the current mode variable in the ctrl struct is updated to indicate that the port is in READ mode, line 22. The port is confiugred in exactly the same way for a WRITE mode, except that here the mode is set for an open drain output.

The choice of using open drain outputs has been made during development to reduce the risk of two outputs driving the same line, as the data-lines are driven by both the FPGA and the MCU. The line is instead driven high by a pull-up resistor. This reduces the speed of the communication as a much larger rise-time is present, but it protects the rather expensive FPGA from damage.

The physcical port is then updated in line 43 and 44 by the function \textbf{HAL\_GPIO\_Init}. This is once again an STM32 specific function that updates the specific microcontroller registers in order to configure the microcontroller for the specified port settings.

\subsection*{SetIOValue}
As the 16 bit wide communication port is a concatenation of two ports, namely pins 0-7 on port B and pins 0-7 on port C the data written to the communication port must first be manipulated in such a way that the 8 least significant bits are set on pins 0-7 of port B, and the 8 most significant bits are set on pins 0-7 on port C.

This function directly manipulates the registers of the STM32F446RE, and as such care is taken not to change any data on the pins of port B and C that are not used for the communication port. All used 8 bits of a port are set at the same time, and as such it must be ensured that non of the unused bits are changed. I.e. only pins 0-7 of port B are used for the communication port, when writing data to these pins, no changes should be made to the data on pins 8-15 on port B, the same can be said for port C.

The code for the \textbf{SetIOValue} function can be seen in listing \refq{lst:7_3_1_SetIOValue}.

\lstinputlisting[language=c ,style = c,firstnumber=1, linerange=1283-1307, caption={Code for the SetIOValue function used to set a value on the 16 bit wide port connection the MCU and Sample Control Module together.}, label={lst:7_3_1_SetIOValue}]{Appendix/Code/main_2612.tex}

the variable lowByte is set to take the 8 least significant bits from the input value, and whatever data is already on pins 8-15 of the port in line 13. The same is done for the 8 most significant bits in line 14. This ensures that whatever data is on bits 8-15 of both port B and C are not changed.

To fully describe this, line 13 is seperated into the different steps within the line. The opperation \textbf{(cp->ctrl.LowBytePort -> ODR) \& 0xFF00} is a bitwise opperation. The complete value present on the port, in this case port B is logically anded with the HEX vale FF00, or 652800, in binary this is 1111 1111 0000 0000. The result is that only the value present on bits 8-15 of port B will be left. 

The operation \textbf{(IOValue\&0xFF)} ANDs the input value with the binary value 0000 0000 1111 1111, resulting in only the 8 least significant bits of the input value being left. The previously loaded value of the unchanged 8 most significant bits already on the port from \textbf{(cp->ctrl.LowBytePort -> ODR) \& 0xFF00}, is then logically ORed with the 8 least significant bits of the input value from \textbf{(IOValue\&0xFF)}. This results in only the 8 least significant bits of port B being updated, ensuring that the data on bits 8-15 of port B are not changed.

If for example the value 1011 1000 1100 0011 is currently on port B, and the communication port should be updated to 1001 0110 1001 0110, the opperation would be as follows:

(1011 1000 1100 0011 \& 1111 1111 0000 0000) | (1001 0110 1001 0110 \& 0000 0000 1111 1111)

This can be done in multiple steps, such that the first step is:

(1011 1000 1100 0011 \& 1111 1111 0000 0000)

Resulting in 1011 1000 0000 0000 

The next step is then: 

(1001 0110 1001 0110 \& 0000 0000 1111 1111) 

Resulting in 0000 0000 1001 0110. 

These two results are then ORed together as:

(1011 1000 0000 0000) | (0000 0000 1001 0110) 

Resulting in 1011 1000 1001 0110. Here the resulting bits 0-7 are updated to the 8 least significant bits of the value that is to be on the communication port, and bits 8-15 of port B are unchanged, as desired. Practically the same operation is done to set the most significant bits of the communication port on port C. The only change here is that the input value to be set on the port is bitshifted left by 8.

Once the variables lowByte and highByte have been found, a sanity check on the ctrl struct is done, the lowBytePort is checked to be port B and highBytePort to be port C. If the sanity check is passed, port B is updated in line 16 and port C in line 17. ODR is the output register of the STM32F446RE, this dirrectly sets the value at the given output port.

\subsection*{GetIOValue}
The function \textbf{GetIOValue} fetches the data present on the communication port and stores it at an allocated address by the use of a pointer. The function can be seen in listing \refq{lst:7_3_1_GetIOValue}.

\lstinputlisting[language=c ,style = c,firstnumber=1, linerange=1309-1326, caption={Code for the GetIOValue function used to get the value on the 16 bit wide port connection the MCU and Sample Control Module together.}, label={lst:7_3_1_GetIOValue}]{Appendix/Code/main_2612.tex}

The function takes in the pointer \textbf{*result}, whatever address this pointer points at is then used to store the value found on the communication port. Functionally all the hardware specified for the communication port is loaded in line 2. line 6 is then a check to ensure that the LowBytePort is port B and HighBytePort is port C. Line 7 reads the 8 most significant bits from port C, hereafter they are anded with 0000 0000 1111 1111, leaving only the 8 least significant bits of port C. The 8 least significant bits read from Port C is then bitshifted by 8, this value is then stored in readVal.

line 8 reads the 8 least significant bits of port B in exaclty the same way port C is read, these bits are however not bitshifted. The value found at port C is then logically ored with the value found at port B. Line 9 then stores the complete value of the communication port at the designated address from the input pointer \textbf{*result}.

\subsection*{Get}
The functions available in the \textbf{Get} struct are those that are generally used to either read data from the communication port or write data to the communication port. The code for the functions will not be described in depth, the functionality of these functions will instead be explained. 

The \textbf{RestComm(void)} function is used to reset the Sample Control Module, see section \refq{subsec:Memory}. This is done in the following way:
\begin{itemize}
    \item CommPort.set.SetIXMode(INTERNAL)
    \item CommPort.set.SetRnW(WRITE)
    \item CommPort.set.PulseCLK()
    \item CommPort.set.SetRnw(READ)
    \item CommPort.set.PulseCLK() x4
\end{itemize}

essentially a single pulse on the clock pin is sent in WRITE mode to ensure that the counter for the reset logic in the Sampel Control Module is set to 0, hereafter 4 consecutive clock pulses are sent in READ mode, ressting the Sample Control Module.

The \textbf{WriteData(uint16\_t data, uint16\_t addr)} function is used to write data to a specific register of the Sample Control Module. This function configures the communication port as an output, it then uses the \textbf{CommPort.set.setIOValue(uint16\_t IOValue)} function to write the designated address to the communication port. Once the address is set, a clock pulse is sent and the \textbf{CommPort.set.setIOValue(uint16\_t IOValue)} function is then used to set the desired data to the data-lines, followed by a clock. 

The \textbf{FetchData(uint16\_t *result, uint16\_t addr)} function is used to read an internal register of the Sample Control Module, i.e. it is used to read the data at an address. This function cannot be used to read sample data, as this is clocked out sequentially once the port is in EXTERNAL mode. In much the same way as the WriteData function, the communication port is first configured as an output and the address of the desired register is sent, followed by a clock pulse. The communication port is then configured as an input using the \textbf{CommPort.set.SetIOMode(enum IO mode)} and \textbf{CommPort.set.SetRnW(enum IO mode)} functions. A clock pulse is then sent and the \textbf{CommPort.set.GetIOValue(uint16\_t *result)} function is then used to fetch the data on the bus. This is then stored at the address of the given input pointer. 

The core functions and general structure of the CommunicationPort data-type has been introduced in this section, the full set of C code for this can be seen in appendix \refq{App:MCUCode}.

%Comm port code etc
\chapter{Problem Statement} \label{ch:ProblemStatement}
There is a gap in the market for LCR meters and impedance analyzers. The current commercial offerings with more advanced features are too expensive for the individual to even consider. Brands like Rigol and Siglent have made instruments like oscilloscopes and signal generators something most people can afford, but the same revolution has not yet been precent for impedance analyzers. One interesting point to make here is that companies such as Rohde \& Schwartz produce hardware capable of \SIQ{10}{\mega\hertz} analysis and market it at the price of 5000 €, LCX-200, just to sell a software upgrade unlocking the full \SIQ{10}{\mega\hertz} bandwidth. In other words, a fully functional \SIQ{10}{\mega\hertz} unit is sold as a \SIQ{500}{\kilo\hertz} unit, indicating that the hardware is not a major price limitation. 

Furthermore the advancement in digital signal processing, speed and price of digital circuitry indicates that the hardware requirements for a high bandwidth LCR meter should be obtainable at a lower price than the current market offers, as is also indicated by the oscilloscopes and signal generators produced by Siglent and Rigol.

Chapter \refq{ch:TechnicalAnalysis} indicates that numerous methods can be used to measure an unknown impedance. Most commercial impedance analyzers however use the auto-balancing bridge topology. At high frequencies this requires extensive analog circuitry to implement a null detector, phase-detector and voltage-phase controlled oscilator. It follows that a large amount of analog circuitry must be costly to both develop and produce. This project instead will focus on the use of the IV-method as this is a somewhat much simpler principle, most notably much less complex analog circuitry is required for the IV-method than the auto-balancing bridge topology.

All This leads to the impression that it must be possible to design impedance analyzers at a fraction of the current market price. This project will seek to bridge the apparent market gap between cost and functionality, and bring the impedance analyzers up to date with other current measurement equipment. To further reduce the hardware pricing, the overall accuracy of the targeted instrument is acceptable to be less than say that of the LCX-200, as the target group is hobbyist. This conclusion eventually leads to the following problem statement,
that this report will try to answer:

\textbf{\textit{How can an impedance analyzer be constructed using the IV-metod, when it must have a similar bandwidth, resolution, accuracy, I/O and ability to display data in a way that is typically only found on high-end instruments such as the R\&S LCX-200?}}
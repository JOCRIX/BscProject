\chapter{Discussion} \label{ch:Discussion}

This was a very ambitious project that aimed to deliver a cost-effective, and capable, impedance analyzer to a hobbyists and small businesses. The initial focus on identifying a market gap provided a clear direction, ensuring that the project addressed the needs of the target group. The hardware development, particularly the sampling system, was a complete success. It effectively meets the core requirement of sampling the voltage and current signals, saving them in memory and allowing a microcontroller to process the samples and make the results available to the user, thus demonstrating that the fundamental design objectives were achieved.

However, not all aspects of the project met the intended specifications. Several software functionalities remain incomplete, primarily due to time constraints rather than technical challenges. These features are critical for improving the analyzer's usability and aligning the project fully with the original requirements and intentions. Additionally, the inability to verify the system's accuracy due to the lack of reference components and test equipment poses a limitation, but one that can be addressed with more time. The performance of the instrument is affected by a lack of calibration functions, specifically proper open circuit and short circuit models for the impedance analyzers front end and probe interface. It can measure en open and short circuit impedance, but this is not fully developed at this time, this significantly affects the instruments accuracy, especially at higher frequencies.

Despite these shortcomings, the robust performance of the hardware provides a solid foundation for future development. With additional resources and time allocated to software improvements and accuracy verification, the impedance analyzer has the potential to evolve into a highly reliable and valuable tool. This project underscores the importance of balancing ambition with practical constraints while laying the groundwork for future enhancements.


It represents a meaningful step toward making advanced electronic testing equipment more accessible.
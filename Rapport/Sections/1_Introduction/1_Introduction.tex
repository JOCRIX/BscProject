\chapter{Introduction} \label{ch:Introduction}
Among hobbyists, tinkerers and smaller companies, there is a need to characterize the most basic one-port electronic devices such as capacitors, inductors and resistors. Resistors can be bought, relatively, cheaply while still having a high degree of accuracy. Inductors and capacitors, however, often have significantly looser tolerances. Another limitation of reactive devices is the level of detail provided in their datasheets. Often values such as ESR or phase angle are not stated, nor their frequency or bias voltage stability. These parameters can be crucial for a designer when engineering an accurate, reliable and predicable circuit.

These component parameters can be measured, or at least estimated by the means of a rather large and complicated test-setup, involving both signal generators and oscilloscopes. The amount of steps required to take measurements such as these makes it prone to errors inherent to the test setup. As an added constraint; the test procedure is tedious to perform and, because of the limited range of the setup, the results can be inaccurate, especially at higher frequencies.

Professional test equipment for exactly this purpose is available on the market and are known in the industry as LCR meters. These devices come in a broad range of measurement capabilities and frequency ranges. Whilst these devices are available, the LCR meters with both good accuracy, flexible capabilities and high frequency ranges are prohibitively expensive and beyond the reach of most individuals, and small companies. These issues leads to the initial problem statement of this project.

“Why are these high-end LCR meters so expensive? Is it possible to estimate the characteristics of a passive device in a more economical way than what is present on the market today?”

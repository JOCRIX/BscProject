\subsection{Reactance} \label{subsec:Reactance}
To find $jX$ is trivial and will be done only for a capacitor, the process for an inductor is the same. Consider the circuit on figure \refq{fig:4_1_1_CapCircuit} with a voltage source, $v(t)$, connected across a capacitor and a current $i(t)$ flowing in the circuit. 
\begin{figure}[H]
    \centering
    \includegraphics[clip, trim=0 400 0 0, width=0.60\textwidth]{Sections/4_TechnicalAnalysis/Figures/4_1_1_CapCircuit.pdf}
    \caption{An AC voltage source, $v(t)$, across a capacitor causes a current, $i(t)$, to flow through it.}
    \label{fig:4_1_1_CapCircuit}
\end{figure}

The current in the circuit is, because of the capacitor, as in eq \refq{eq:4_1_1_CapCurrent}.
\begin{equation}\label{eq:4_1_1_CapCurrent}
    i(t) = C\cdot\frac{d}{dt} (v(t))
\end{equation}
Transforming \refq{eq:4_1_1_CapCurrent} into the frequency domain with the laplace transform gives eq \refq{eq:4_1_1_CapCurrent2}.
\begin{equation}\label{eq:4_1_1_CapCurrent2}
    \Laplace\{\/i(t)\}\ = \Laplace\{\/C\cdot\frac{d}{dt} (v(t))\} \rightarrow I(s) = C\cdot(sV(s) - v(t=0^-))
\end{equation}
The circuit is supplied by an AC source and it is assumed that the circuit has reached a steady-state so that the initial condition of the voltage source is $v(t=0^-) = 0$. Solving for the reactance of the circuit $V(s) / I(s)$ gives eq \refq{eq:4_1_1_CapCurrent3}.
\begin{equation}\label{eq:4_1_1_CapCurrent3}
    \frac{V(s)}{I(s)} =X_c(s)= \frac{1}{sC}
\end{equation}
Substituting the laplace variable $s$ with $s = 0 + j\omega$ gives eq \refq{eq:4_1_1_CapCurrent4}.
\begin{equation}\label{eq:4_1_1_CapCurrent4}
    X_c(\omega) = \frac{1}{j\omega C}=-\frac{j}{\omega C}
\end{equation}
Note that the complex laplace variable is defined as $s = \sigma + j\omega$. The real part $\sigma$ represents exponential growth. By assuming the circuit is in steady-state, there is no exponential growth so $\sigma = 0$. The impedance for a circuit dominated by a capacitance will have the form $Z = R -\frac{j}{\omega C}$ as shown in eq \refq{eq:4_1_1_CapCurrent4}.

The value of capacitance can be found with eq \refq{eq:4_1_1_CapCurrent4} by just solving for C. This gives the phasor in eq \ref{eq:4_1_1_CapCurrent5}
\begin{equation}\label{eq:4_1_1_CapCurrent5}
    C(\omega)=-\frac{j}{X_C\cdot \omega}
\end{equation}
The quantity of interest at this point is the magnitude of capacitance as shown in eq \refq{eq:4_1_1_CapCurrent6}.
\begin{equation}\label{eq:4_1_1_CapCurrent6}
    |C(\omega)|= \sqrt{\left(-\frac{1}{X_C\cdot \omega}\right)^2} =\frac{1}{X_c\cdot \omega} 
\end{equation}

The result in eq \refq{eq:4_1_1_CapCurrent5} is obvious as the phasor in eq \refq{eq:4_1_1_CapCurrent5} is just pointing down the imaginary axis in the complex plane and thus has a magnitude of $abs(C)$. A similar exercise can be performed if the capacitor on figure \refq{fig:4_1_1_CapCircuit} was swapped with an inductor. The results is the expression for inductive reactance in eq \refq{eq:4_1_1_IndReactance}.
\begin{equation}\label{eq:4_1_1_IndReactance}
    X_l(\omega) = j \omega L
\end{equation}

Much like eq \refq{eq:4_1_1_CapCurrent6} the inductance value can be found with eq \refq{eq:4_1_1_IndReactance}. This is done in eq \refq{eq:4_1_1_IndReactance2}.

\begin{equation}\label{eq:4_1_1_IndReactance2}
    L(\omega) = \frac{X_l(\omega)}{j\omega} = -j\frac{X_l(\omega)}{\omega}
\end{equation}

Similar to eq \refq{eq:4_1_1_CapCurrent6} the magnitude of inductance is found and is as shown in eq \refq{eq:4_1_1_IndReactance3}.

\begin{equation}\label{eq:4_1_1_IndReactance3}
    |L(\omega)| = \sqrt{\left(-\frac{X_l(\omega)}{\omega}\right)^2} = \frac{X_l(\omega)}{\omega}
\end{equation}
\section{Requirement Review} \label{subsec:8_1_ReqReview}

All of the requirements from table \refq{tab:5_SystemRequirements} have been reviewed in order to see if the product meets the specifications.


\subsubsection*{Requirement §2: Frequency range}
The system must be capable of generating sine waves from \SIQ{50}{\hertz} to \SIQ{1}{\mega\hertz} with a resolution of \SIQ{1}{\hertz}.

\textbf{Status:} The DAC is controlled by a DDS block in the Sample Control Module, at a sample rate of \SIQ{20}{\mega\hertz}. The DDS block has a resolution of \SIQ{4.657}{\milli\hertz}, essentially allowing the DAC to generate frequencies down to \SIQ{4.657}{\milli\hertz}. The reconstruction filter however is AC coupled with a simulated lower cut-off of less than \SIQ{10}{\hertz}. The upper cut-off frequency for the reconstruction filter is simulated to more than \SIQ{1.5}{\mega\hertz}. This requirement is deemed ensured by desgin, but has not been tested explicitly. The output have been observed to generate a pure sine wave at \SIQ{1}{\mega\hertz} with an oscilloscope.
\nl
\nl

\subsubsection*{Requirement §2: Impedance Fit}
Impedance fit to eiter series model or parallel model.

\textbf{Status:} This is implemented in software, the functionality is shown in section \refq{sec:MCU}.
\nl
\nl

\subsubsection*{Requirement §3: Impedance Range}
The instrument should be able to measure an impedance in the interval $10m\Omega < |Z| < 100M \Omega$. 

\textbf{Status:} This has been tested and the results can be seen in appendix \refq{App:AccuracyBWTest}. The test results are very promising, but ultimately inconlusive. The proper test equipment was not available to perform this test.
\nl
\nl

\subsubsection*{Requirement 4 and 5: Modulus and Argument accuracy}
The specific requirement can be seen in section \refq{sec:ModulusArgumentAccuracy}.

\textbf{Status:} The instrument accuracy has been tested at varies impedances and frequencies, no higher than \SIQ{300}{\kilo\hertz}, these tests can be seen in appendix \refq{App:AccuracyBWTest}. A review of this can be seen in section \refq{subsec:8_1_AccTest}. These requirements are not met.
\nl
\nl

\subsubsection*{Requirement 6: Test voltage Range}
It must be possible to set the test voltage in the range of \SIQ{10}{\milli\volt}$_{pp}$ to \SIQ{10}{\milli\volt}$_{pp}$ with a resolution of \SIQ{10}{\milli\volt}.

\textbf{Status:} It is only possible to set the test voltage at 4 fixed levels, \SIQ{0.6}{\volt}$_{pp}$, \SIQ{1.2}{\volt}$_{pp}$, \SIQ{2.3}{\volt}$_{pp}$ and \SIQ{4.3}{\volt}$_{pp}$. This requirement is not met.
\nl
\nl

\subsubsection*{Requirement 7: Maximum Test Current}
The system must be capable of generating a minimum \SIQ{100}{\milli\ampere}$_{pp}$.

\textbf{Status:} It is only possible to set the at 4 fixed levels, \SIQ{0.6}{\volt}$_{pp}$, \SIQ{1.2}{\volt}$_{pp}$, \SIQ{2.3}{\volt}$_{pp}$ and \SIQ{4.3}{\volt}$_{pp}$. This requirement is not met.
\nl
\nl

Requirement §7 states that the maximum test current should be $100mApp$ this requirement is not met.

Requirement §8 states that the instrument should have an adjustable DC bias voltage for the DUT. This was not implemented due to time constraints.

Requirement §9 states that the instrument should be able to calculate the DUT parameters such as L,C,R,D,Q and so on. The instrument can do this and display them to the user.

Requirement §10, §11, §12, and §13 says that the instrument should have a sweep capability. The instrument can set the test frequencies from software, but the functionality to let it sweep this parameter has not been implemented due to time constraints.

Requirement §14, §15 and §16 was not implemented due to time constraints.

Requirement §17 states that the insturment should have a 7" touch screen. It does have a touchscreen, but the software for this is not fully developed.

Requirement §18 states that the instrument should have a 4-wire kelvin interface with the DUT. This is ensured by design.

Requirement §19 stateas that the instrument should be able to export data over UART. It can do this, as has been demonstrated.


\section{Problem Analysis Conclusion} \label{sec:probAnalConc}
The impedance of a DUT can be measured with a complicated setup using oscilloscopes and signal generators, but this setup is prone to human errors and inaccuracies that are caused by, potentially, long test probe leads, probe and oscilloscope input capacitances and so on. The measurement result must then be manually converted into the quantities the user is interested in, in order to analyze the DUT. This is both cumbersome, time consuming and somewhat inaccurate.

There exists a plethora of commercial instruments to perform, and automate, a lot of the same measurements that would otherwise have to be done manually. They vary wildly in their capabilities and in their price. The price is often a show-stopper for many individual electronic developers when they are considering acquiring a more capable instrument. There is a demand in the market for a more capable LCR meter, or impedance analyzer, with greater range and capability at a lower price point.

One important thing to note here is that the general tendencies for the lower priced impedance analyzers and LCR meters is that their bandwidth typically stops at \SIQ{100}{\kilo\hertz} or \SIQ{300}{\kilo\hertz}. The price dramatically increases for higher frequency ranges. The focus thus lies on the LCR meters in the frequency range up to about \SIQ{1}{\mega\hertz}, as these are the ones with the highest costs, making component analysis at \SIQ{300}{\kilo\hertz} to \SIQ{1}{\mega\hertz} practically unobtainable for the average hobbyist.
\section{Market demand} \label{sec:ProfessionalMarketDemand}
Generally there is no shortage of LCR meters and impedance analysers. Manufactures such as Keysight produce the E4982A LCR meter,
 capable of measuring up to \SI{500}{\mega\hertz}, or the LCX-200 from Rohde \& Schwarz capable of up to \SI{10}{\mega\hertz}
analysis. Many other manufacturers produce instruments with similar capabilities, but the price tag drastically increases with frequency
and as previously mentioned a \SI{500}{\kilo\hertz} LCX-200 cost 15000 €. 

Forums for electronics hobbyists such as Eevblog or a more general one such as Reddit both have topics regarding the price of LCR
meters, as well as the need for them, eevblog \cite{EevblogLCR} and Reddit \cite{RedditLCR}. One area where the need for components
characterization is vital is switching converts. The switching frequencies of converters is increasing to the megahertz region, and 
many components only have ESR or phase angle values stated at a single frequency in their datasheet, if it is even stated at all.

The writers of this document have both been in situations where component insight is critical to a project, this includes class D
audio amplifiers output filters and induction heaters main coil. Here elements such as loss factor or Q are crucial to the performance
of the system/project, and as these projects have been on a hobby level, buying an expensive component analyser is simply not feasible.
This conclusion is also supported by the different eevblog and reddit posts with regards to expensive LCR meters.

Buying an instrument from a company like Keysight provides a business with a certain confidence in their measurements. The business may have a lot of expertise in the instruments and softwares that come from a particular instrument manufacturer, along with a history of being a known \textit{good} manufacturer of instruments. Despite this it is possible to have a sustainable business where a significant amount of the revenue comes from small businesses and individuels that don't place the same weight on those parameters. Rigol Technologies and Siglent have demonstrated this with their entry level oscilloscopes, signal generators and multimeters that have become common place in many home \textit{labs}. 

In short, there seems to be a "gab" in the market, where hobbyist have a need or desire for component analysis, but the available
products are simply too expensive. When comparing recent market development of oscilloscopes, signal generators and multimeters,
it seems the component analysis area just is not keeping up, the conclusion is simple, it seems that there is a desire and need for component analysis, but due to the pricing of commercial 
impedance analyzers, hobbyists must find alternative methods such as mentioned in section \ref*{sec:MeasureReactiveComponents}. 
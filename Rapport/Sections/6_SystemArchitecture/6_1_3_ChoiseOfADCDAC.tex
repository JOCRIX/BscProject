\subsection{Choice of ADCs and DAC} \label{subsec:ADC_DAC_Choice}
The System requirements specify that the best case scenario accuracy of impedance magnitude must be less than \SIQ{0.1}{\%}. Impedance will be calculated as voltage divided by current, with both voltage and current being sampled by their own ADC. Assuming a worst case scenario where the error of one ADC directly adds to that of the other, such that the total ADC error can be given as $E_{ADC1}+E_{ADC2} = E_{ADC}$, the total ADC error $E_{ADC}$ should be less than the required magnitude accuracy, such that each ADC should be accurate to \SIQ{0.05}{\%}.

To ensure that the ADCs are accurate it has been chosen to use two 16 bit ADCs, more specifically two LTC2311 \cite{ADC_LTC2311}. These are 16 bit analog to digital converters with a maximum sample rate of 5 Msps. These offer a guaranteed integral non-linearity (INL) of less than 8 lsb, i.e. less than \SIQ{0.0122}{\%}, resulting in a guaranteed maximum ADC error $E_{ADC}$ of less than \SIQ{0.0244}{\%}.

This error assumes that \SIQ{100}{\%} of the ADCs range is utilized, this is of course not always the case. To find the minimum range the ADCs can be used within, the 8 LSB non-linearity can be considered as the maximum allowed \SIQ{0.05}{\%} error. This can be seen in equation \ref{eq:6_1_3_ADC_range_error}. Here it is also shown that the ADCs can be used down to only \SIQ{24.4}{\%} of their range. This ensures that even if the ADCs full range is not used, the non-linearity at least should not add more error than tolerated.

\begin{equation}
\begin{split}
    \label{eq:6_1_3_ADC_range_error}
    n_{range} &= \frac{INL}{Error_{MAX}} \cdot 100 \\
    \Rightarrow n_{range} &= \frac{8}{0.05} \cdot 100 \\
    \Rightarrow n_{range} & = 16000 \Rightarrow \frac{16000}{65535}\cdot 100 = \SIQ{24.4}{\%}
\end{split}
\end{equation}

Likewise a 16 bit DAC have been chosen, LTC1668 \cite{DAC_LTC1668}. This DAC is capable of 50 Msps, ensuring good spectral purity even at \SIQ{1}{\mega\hertz}.
\chapter{LaTex guide}
%%%%%%%%%%%%%%%%%%%%%%%%%%%
Til tabeller:
https://www.tablesgenerator.com/

%%%%%%%%%%%%%%%%%%%%%%%%%%%
PNG to PDF:
https://png2pdf.com/

%%%%%%%%%%%%%%%%%%%%%%%%%%%
Referencer:
Kapitel: \label{ch:__}
Section: \label{sec:__}
Subsection: \label{subsec:___}
Subsubsection: \label{subsubsec:___}
Equation: \label{eq:Chapter_name}
Figure: \label{fig:Chapter_name}
Tabel: \label{tab:Chapter_name}
Appendix: \label{app:___}


Referencer med flere ord skrives sådan:
\label{ch:En_Stor_Ko}
%%%%%%%%%%%%%%%%%%%%%%%%%%%
Til kommentarer:
\todo{Dette er en kommentar}

%%%%%%%%%%%%%%%%%%%%%%%%%%%
Enhed:
\SI{}{}
%Her skrives det de fulde ord, så eksempelvis \SI{30}{\centi\meter} eller \SI{12}{\milli\Ohm}

%%%%%%%%%%%%%%%%%%%%%%%%%%%
For at sætte en figur ind:
%\begin{figure}[H]
 %   \begin{adjustwidth}{}{}
  %  \centering
   % \includegraphics[trim = 0 0 0 0, clip, width=1\textwidth]{figures/Mappe/figur.pdf}
 %   \end{adjustwidth}
  %  \caption{SKriv noget text der forklarer figuren..}
   % \label{fig:label_til_figur - Den her kan man \ref til}
%\end{figure}
Trim kan trimme billedet :)
trim = left lower right upper

%%%%%%%%%%%%%%%%%%%%%%%%%%%
Til lister
\begin{itemize}
  \item List entries start with the \verb|\item| command.
  \item Individual entries are indicated with a black dot, a so-called bullet.
  \item The text in the entries may be of any length.
\end{itemize}
%%%%%%%%%%%%%%%%%%%%%%%%%%%
Til Ligninger
\noindent\begin{minipage}{.5\linewidth}
\begin{equation*}
    \label{eq:Band_Pass}
    H_{BandPass}(s)=a\cdot\frac{(s-z)}{(s-p_1)\cdot(s-p_2)}
\end{equation*}
\end{minipage}%
\begin{minipage}{.4\linewidth}
\begin{flalign*}
a &= gain \\
 s& =j\cdot\omega\\
\omega&=2 \cdot \pi \cdot f
\end{flalign*}
\end{minipage}
\begin{minipage}{.1\linewidth}
\ref{eq:Band_Pass}
\end{minipage}


For flere equations:

For at tilfredsstille Emil:
\begin{flalign}
    ligning 1\\[20pt]
    ligning 2\\[20pt]
    ligning 3
\end{flalign}

Flere equations med kun 1 label (de alligner ved \&):
\begin{equation}
    \begin{aligned}
    \label{indsæt label her}
        ligning & 1\\[20pt]
        ligning & 2\\[20pt]
        ligning & 3
    \end{aligned}
\end{equation}

%%%%%%%%%%%%%%%%%%%%%%%%%%%
Indsæt pdf i appendix:
Har lavet en macro:
%usepackage{pdfpages}

%\pdfappendix[TITEL]{LABEL}{PDF LOKATION}

%%%%%%%%%%%%%%%%%%%%%%%%%%%
%\Chap{TITEL}{LABEL}
\Sec{TITEL}{LABEL}
\Sub{TITEL}{LABEL}
\Subsub{TITEL}{LABEL}

Sådan indsætter du både et nyt afsnit med label.


%%%%%%%%%%%%%%%%%%%%%%%%%%%%
% Eksempel på kode indsæt
% https://www.overleaf.com/learn/latex/Code_listing

% Der er i preabmle lavet en styling for koden, hvor man kan ændre farve og så vidre...

%%%
% Ren kode eksempel

\begin{lstlisting}[language=C++, caption=WiFi connection]
// Include library.
#include <WiFi.h>

// Define the wanted connection.
const char* ssid = "Test_SW";                  
const char* password = "SmartWatch";

void setup(){
// Establish connection.
WiFi.begin(ssid, password);
}
\end{lstlisting}
%%%
% Dokument kode eksempel

%\lstinputlisting[language=Octave, firstline=2, lastline=12]{BitXorMatrix.m}

% Her "linker" vi til det dokument som vi vil vise i den anden tuborg-klæmme. Og first line og lastline er den linje vi starter ved og den vi slutter på tilsvarrende. 

%Overstregning af tekst
%\colorbox{1}{2}  %1: Farven til overstregning (SKAL skrives med småt. fx red, yellow, green, blue...Alle farverne kan ses her: https://www.overleaf.com/learn/latex/Using_colours_in_LaTeX)  %2: Teksten der skal overstreges.
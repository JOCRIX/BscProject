\selectlanguage{english}
\pdfbookmark[0]{Danish title page}{label:titlepage_en}

\aautitlepage{%
  \englishprojectinfo{
     %title
     Design of a Spectral Impedance Analyzer for Passive Elements
  }{%
    Designing and constructing an Impedance Analyzer%theme
  }{%
    Fall Semester 2024 %project period
  }{%
    Group 771 - EIT7% project group
  }{%
    %list of group members
    Joachim R.B. Andersen\\
    Jakob F. Thiesen
    
  }{%
    %list of supervisors
    %The omnissiah
    Troels B. Sørensen
  }{%
    2 % number of printed copies
  }
  { %number of pages in the main matter
   A lot
  }
  {%
    \today % date of completion
  }%
}{%department and address
  \textbf{Electronics and IT}\\
  Aalborg University\\
  \href{http://www.aau.dk}{http://www.aau.dk}
}{% the abstract
\small
This project focuses on the analysis and development of a cost-effective impedance analyzer specifically desgined for characterization of passive components such as capacitors, resistors and inductors. \\

A problem analysis concludes that the reduction in cost of test equipment seen in recent years, such as for oscilloscopes, is not seen for impedance analyzers in the \SIQ{1}{\mega\hertz} range, making these instruments practically inaccessible for the broader audience, such as hobbyists, educational institutions, and small-scale enterprises. \\

This project aims to fill in the gap between entry level LCR meters and expensive high-end impedance analyzers by developing an accurate high frequency impedance analyzer. This is done through the IV-method and precise synchoniced sampling and spectral analysis, rather than the typically used auto-balancing bridge topology, removing both cost and complexity of the analog circuitry. \\

The proposed design has been realized and implemented on hardware showing good results for accurate impedance analysis at a reduced cost. This project offers a practical solution for affordable impedance measurement as well as contributing to making advanced electronic testing equipment more accessible to the general public.
\normalsize
}

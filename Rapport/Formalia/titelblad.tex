\selectlanguage{english}
\pdfbookmark[0]{Danish title page}{label:titlepage_en}

\aautitlepage{%
  \englishprojectinfo{
     %title
     Design of a Spectral Impedance Analyzer for Passive Elements
  }{%
    Designing and constructing an Impedance Analyzer%theme
  }{%
    Fall Semester 2024 %project period
  }{%
    Group 771 - EIT7% project group
  }{%
    %list of group members
    Joachim R.B. Andersen\\
    Jakob F. Thiesen
    
  }{%
    %list of supervisors
    %The omnissiah
    Troels B. Sørensen
  }{%
    2 % number of printed copies
  }
  { %number of pages in the main matter
   A lot
  }
  {%
    \today % date of completion
  }%
}{%department and address
  \textbf{Electronics and IT}\\
  Aalborg University\\
  \href{http://www.aau.dk}{http://www.aau.dk}
}{% the abstract
This project focused on the development of a cost-effective impedance analyzer specifically designed for the characterization of passive components such as capacitors and inductors.

The initial phase involved a problem analysis to identify the target market, assess existing impedance analyzers in terms of functionality, cost, and accessibility. Our analysis revealed a gap in the market for a device that combines key features of high-end instruments at a price point accessible to a broader audience, including hobbyists, educational institutions, and small-scale enterprises.

Based on this, we developed an impedance analyzer with specifications aimed at delivering high performance at a reduced cost. The design phase included defining critical system requirements, followed by a lengthy hardware, and software, design phase and culminating in the creation of a PCB. The PCB was subsequently tested to validate the projects functionality against the set specifications.

The resulting impedance analyzer demonstrated capabilities for measuring impedance across a wide frequency range and is suitable for characterizing important parameters of passive components. This project offers a practical solution for affordable impedance measurement but also contributes to making advanced electronic testing equipment more accessible to the general public.
}
